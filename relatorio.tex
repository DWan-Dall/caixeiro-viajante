% Options for packages loaded elsewhere
\PassOptionsToPackage{unicode}{hyperref}
\PassOptionsToPackage{hyphens}{url}
%
\documentclass[
]{article}
\usepackage{amsmath,amssymb}
\usepackage{lmodern}
\usepackage{iftex}
\ifPDFTeX
  \usepackage[T1]{fontenc}
  \usepackage[utf8]{inputenc}
  \usepackage{textcomp} % provide euro and other symbols
\else % if luatex or xetex
  \usepackage{unicode-math}
  \defaultfontfeatures{Scale=MatchLowercase}
  \defaultfontfeatures[\rmfamily]{Ligatures=TeX,Scale=1}
\fi
% Use upquote if available, for straight quotes in verbatim environments
\IfFileExists{upquote.sty}{\usepackage{upquote}}{}
\IfFileExists{microtype.sty}{% use microtype if available
  \usepackage[]{microtype}
  \UseMicrotypeSet[protrusion]{basicmath} % disable protrusion for tt fonts
}{}
\makeatletter
\@ifundefined{KOMAClassName}{% if non-KOMA class
  \IfFileExists{parskip.sty}{%
    \usepackage{parskip}
  }{% else
    \setlength{\parindent}{0pt}
    \setlength{\parskip}{6pt plus 2pt minus 1pt}}
}{% if KOMA class
  \KOMAoptions{parskip=half}}
\makeatother
\usepackage{xcolor}
\IfFileExists{xurl.sty}{\usepackage{xurl}}{} % add URL line breaks if available
\IfFileExists{bookmark.sty}{\usepackage{bookmark}}{\usepackage{hyperref}}
\hypersetup{
  hidelinks,
  pdfcreator={LaTeX via pandoc}}
\urlstyle{same} % disable monospaced font for URLs
\usepackage[margin=1in]{geometry}
\usepackage{longtable,booktabs,array}
\usepackage{calc} % for calculating minipage widths
% Correct order of tables after \paragraph or \subparagraph
\usepackage{etoolbox}
\makeatletter
\patchcmd\longtable{\par}{\if@noskipsec\mbox{}\fi\par}{}{}
\makeatother
% Allow footnotes in longtable head/foot
\IfFileExists{footnotehyper.sty}{\usepackage{footnotehyper}}{\usepackage{footnote}}
\makesavenoteenv{longtable}
\usepackage{graphicx}
\makeatletter
\def\maxwidth{\ifdim\Gin@nat@width>\linewidth\linewidth\else\Gin@nat@width\fi}
\def\maxheight{\ifdim\Gin@nat@height>\textheight\textheight\else\Gin@nat@height\fi}
\makeatother
% Scale images if necessary, so that they will not overflow the page
% margins by default, and it is still possible to overwrite the defaults
% using explicit options in \includegraphics[width, height, ...]{}
\setkeys{Gin}{width=\maxwidth,height=\maxheight,keepaspectratio}
% Set default figure placement to htbp
\makeatletter
\def\fps@figure{htbp}
\makeatother
\setlength{\emergencystretch}{3em} % prevent overfull lines
\providecommand{\tightlist}{%
  \setlength{\itemsep}{0pt}\setlength{\parskip}{0pt}}
\setcounter{secnumdepth}{-\maxdimen} % remove section numbering
\ifLuaTeX
  \usepackage{selnolig}  % disable illegal ligatures
\fi

\author{}
\date{\vspace{-2.5em}}

\begin{document}

\hypertarget{relatuxf3rio-anuxe1lise-do-problema-do-caixeiro-viajante-com-foruxe7a-bruta-e-heuruxedstica}{%
\section{Relatório: Análise do Problema do Caixeiro Viajante com Força
Bruta e
Heurística}\label{relatuxf3rio-anuxe1lise-do-problema-do-caixeiro-viajante-com-foruxe7a-bruta-e-heuruxedstica}}


\hypertarget{autoria}{%
\subsubsection{✍️ Projeto pode ser consultado em:}\label{autoria}}

\href{https://github.com/DWan-Dall/caixeiro-viajante}{Repositório Caixeiro-Viajante - DWan-Dall}��.

Projeto desenvolvido para o Mestrado Profissional em Computação Aplicada
- UNIVALI

Matéria: ANÁLISE DE ALGORITMOS Professor: 
- Rudimar Luiz Scaranto Dazzi

\hypertarget{introduuxe7uxe3o}{%
\subsection{1. Introdução}\label{introduuxe7uxe3o}}

O Problema do Caixeiro Viajante (Traveling Salesman Problem - TSP)
consiste em encontrar o caminho mais curto que permite visitar um
conjunto de cidades exatamente uma vez e retornar à cidade de origem.

\hypertarget{algoritmos-utilizados}{%
\subsection{2. Algoritmos Utilizados}\label{algoritmos-utilizados}}

Neste trabalho, foram implementadas duas abordagens para resolver o TSP:

\begin{itemize}
\tightlist
\item
  \textbf{Força Bruta:} testa todos os caminhos possíveis para encontrar
  a melhor opção.

  \begin{itemize}
  \tightlist
  \item
    Gera todas as possíveis permutações das cidades;
  \item
    Calcula a distância total de cada caminho;
  \end{itemize}
\item
  \textbf{Heurística Escolhida: Vizinho Mais Próximo} - pois aproxima
  uma solução em tempo muito mais rápido.

  \begin{itemize}
  \tightlist
  \item
    Começa em uma cidade;
  \item
    Sempre escolhe a cidade mais próxima não visitada;
  \end{itemize}
\end{itemize}

\hypertarget{metodologia}{%
\subsection{3. Metodologia}\label{metodologia}}

Foram geradas coordenadas aleatórias (x, y) para representar cidades,
utilizando amostras de 5 a 8 cidades.

\begin{itemize}
\tightlist
\item
  Número de cidades variando de 5 a 8.
\item
  Coordenadas geradas aleatoriamente no intervalo {[}0, 100{]}.
\item
  Medição dos tempos de execução e distâncias totais percorridas.
\item
  Gráficos gerados para comparar desempenhos.
\end{itemize}

\hypertarget{ambiente-de-desenvolvimento}{%
\subsection{3.1 Ambiente de
Desenvolvimento}\label{ambiente-de-desenvolvimento}}

\begin{itemize}
\tightlist
\item
  Linguagem: R
\item
  Pacotes:

  \begin{itemize}
  \tightlist
  \item
    dplyr → manipulação de dados;
  \item
    ggplot2 → criação de gráficos;
  \item
    combinat → gerar todas as permutações (para a força bruta);
  \end{itemize}
\item
  Relatório gerado em RMarkdown.
\end{itemize}

\hypertarget{resultados-obtidos}{%
\subsection{4. Resultados Obtidos:}\label{resultados-obtidos}}

\begin{longtable}[]{@{}
  >{\raggedright\arraybackslash}p{(\columnwidth - 8\tabcolsep) * \real{0.2000}}
  >{\raggedright\arraybackslash}p{(\columnwidth - 8\tabcolsep) * \real{0.2000}}
  >{\raggedright\arraybackslash}p{(\columnwidth - 8\tabcolsep) * \real{0.2000}}
  >{\raggedright\arraybackslash}p{(\columnwidth - 8\tabcolsep) * \real{0.2000}}
  >{\raggedright\arraybackslash}p{(\columnwidth - 8\tabcolsep) * \real{0.2000}}@{}}
\toprule
\begin{minipage}[b]{\linewidth}\raggedright
Cidades
\end{minipage} & \begin{minipage}[b]{\linewidth}\raggedright
Distância Bruta
\end{minipage} & \begin{minipage}[b]{\linewidth}\raggedright
Tempo Bruta (segundos)
\end{minipage} & \begin{minipage}[b]{\linewidth}\raggedright
Distância Heurística
\end{minipage} & \begin{minipage}[b]{\linewidth}\raggedright
Tempo Heurística
\end{minipage} \\
\midrule
\endhead
5 & 196 & 0.027 & 200 & 0.0005 \\
6 & 209 & 0.144 & 237 & 0.0006 \\
7 & 229 & 1.17 & 239 & 0.0008 \\
8 & 233 & 10.5 & 244 & 0.001 \\
\bottomrule
\end{longtable}

\begin{itemize}
\tightlist
\item
  Gráfico de Comparativo de Distâncias e Comparação de tempo (com escala
  logarítmica), salvo em
  resultados/grafico-resultado-outras-cidades.png.
\end{itemize}

\hypertarget{anuxe1lise-de-complexidade}{%
\subsection{5. Análise de
Complexidade}\label{anuxe1lise-de-complexidade}}

\begin{itemize}
\tightlist
\item
  O gráfico logarítmico mostra que a força bruta cresce muito mais
  rapidamente que a heurística.
\item
  Confirma o comportamento esperado de O(n!) versus O(n²).
\end{itemize}

\hypertarget{foruxe7a-bruta}{%
\subsection{5.1 Força Bruta}\label{foruxe7a-bruta}}

\begin{itemize}
\tightlist
\item
  Complexidade: O(n!)
\item
  Explicação: Para n cidades, existem n! permutações possíveis de
  caminhos.
\end{itemize}

\textbf{Observação:} Muito ineficiente para número elevado de cidades; o
tempo cresce exponencialmente.

\hypertarget{heuruxedstica-do-vizinho-mais-pruxf3ximo}{%
\subsection{5.2 Heurística do Vizinho Mais
Próximo}\label{heuruxedstica-do-vizinho-mais-pruxf3ximo}}

\begin{itemize}
\tightlist
\item
  Complexidade: O(n\^{}2)
\item
  Explicação: Para cada cidade, busca-se a cidade mais próxima entre as
  restantes.
\end{itemize}

\textbf{Observação:} Rápido e eficiente para tamanhos moderados de
instâncias, mas não garante solução ótima.

\hypertarget{conclusuxe3o}{%
\subsection{6. Conclusão}\label{conclusuxe3o}}

A abordagem de força bruta garante encontrar a rota ótima, mas é
inviável para quantidades maiores de cidades devido ao crescimento
exponencial do tempo de execução. A heurística do vizinho mais próximo
provém soluções aproximadas muito mais rápidas, sendo adequada para
problemas de tamanho médio ou quando é necessário sacrificar um pouco de
qualidade pela rapidez.

O estudo demonstra claramente a diferença prática entre a busca
exaustiva e heurísticas em problemas combinatórios.

A solução por força bruta, embora ótima, torna-se inviável para números
maiores de cidades devido à sua complexidade fatorial. A heurística do
vizinho mais próximo oferece soluções rápidas, ainda que não
necessariamente ótimas, sendo prática para aplicações reais.

\hypertarget{anexos}{%
\subsection{7. Anexos}\label{anexos}}

Script R: caixeiro\_viajante.Rmd

Gráficos: - Comparação de distâncias:
resultados/grafico-resultado-outras-cidades.png - Comparação de tempos:
resultados/grafico-tempo.png

Arquivos CSV - resultados/resultados\_cidades.csv

\section*{Referências Bibliográficas}

\begin{itemize}
    \item R-bloggers. \textit{Travelling Salesman Problem using R}. Disponível em: \url{https://www.r-bloggers.com/2016/12/travelling-salesman-problem-using-r/}. Acesso em abril de 2025.
    
    \item GeeksforGeeks. \textit{Travelling Salesman Problem (TSP) | Implementation}. Disponível em: \url{https://www.geeksforgeeks.org/travelling-salesman-problem-set-1/}. Acesso em abril de 2025.
    
    \item Wikipedia. \textit{Travelling Salesman Problem}. Disponível em: \url{https://en.wikipedia.org/wiki/Travelling_salesman_problem}. Acesso em abril de 2025.
\end{itemize}


\end{document}
